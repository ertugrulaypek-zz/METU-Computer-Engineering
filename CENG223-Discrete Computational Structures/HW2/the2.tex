\documentclass[12pt]{article}
\usepackage[utf8]{inputenc}
\usepackage{float}
\usepackage{amsmath}


\usepackage[hmargin=3cm,vmargin=6.0cm]{geometry}
%\topmargin=0cm
\topmargin=-2cm
\addtolength{\textheight}{6.5cm}
\addtolength{\textwidth}{2.0cm}
%\setlength{\leftmargin}{-5cm}
\setlength{\oddsidemargin}{0.0cm}
\setlength{\evensidemargin}{0.0cm}

%misc libraries goes here


\begin{document}

\section*{Student Information } 
%Write your full name and id number between the colon and newline
%Put one empty space character after colon and before newline
Full Name : ERTUĞRUL AYPEK \\
Id Number : 2171270 \\

% Write your answers below the section tags
\section*{Answer 1}
\begin{table}[H]
\small
\centering
\caption{ Membership Table for question 1.a }

\begin{tabular}{|c|c|c|c|c|c|c|c|c|}	%% specify column number
\hline 							%% line draw
\textbf{A} & \textbf{B} & \textbf{$\overline{A}$} & \textbf{$\overline{B}$} & \textbf{A $\cap$ B} & \textbf{A $\cup$ $\overline{B}$} & \textbf{$\overline{A}$ $\cup$ B} & \textbf{(A $\cup$ $\overline{B}$) $\cap$ ($\overline{A}$ $\cup$ B)} & \textbf{A $\cap$ B $\subseteq$ (A $\cup$ $\overline{B}$) $\cap$ ($\overline{A}$ $\cup$ B)}\\
\hline 
 
1 & 1 & 0 & 0 & 1 & 1 & 1 & 1 & 1 \\ \hline 
1 & 0 & 0 & 1 & 0 & 1 & 0 & 0 & 1 \\ \hline
0 & 1 & 1 & 0 & 0 & 0 & 1 & 0 & 1 \\ \hline
0 & 0 & 1 & 1 & 0 & 1 & 1 & 1 & 1 \\
\hline 

\end{tabular}
\end{table}

As shown in the above table, the statement "(A $\cap$ B)$\subseteq$(A $\cup$ $\overline{B}$) $\cap$ ($\overline{A}$ $\cup$ B)" is true.
\\
\\
\\
 
\begin{table}[H]
\small
\centering
\caption{ Membership Table for question 1.b }

\begin{tabular}{|c|c|c|c|c|c|c|c|c|}	%% specify column number
\hline 							%% line draw
\textbf{A} & \textbf{B} & \textbf{$\overline{A}$} & \textbf{$\overline{B}$} & \textbf{$\overline{A}$ $\cap$ $\overline{B}$} & \textbf{A $\cup$ $\overline{B}$} & \textbf{$\overline{A}$ $\cup$ B} & \textbf{(A $\cup$ $\overline{B}$) $\cap$ ($\overline{A}$ $\cup$ B)} & \textbf{$\overline{A}$ $\cap$ $\overline{B}$ $\subseteq$(A $\cup$ $\overline{B}$) $\cap$ ($\overline{A}$ $\cup$ B)}\\
\hline 
 
1 & 1 & 0 & 0 & 0 & 1 & 1 & 1 & 1 \\ \hline 
1 & 0 & 0 & 1 & 0 & 1 & 0 & 0 & 1 \\ \hline
0 & 1 & 1 & 0 & 0 & 0 & 1 & 0 & 1 \\ \hline
0 & 0 & 1 & 1 & 1 & 1 & 1 & 1 & 1 \\
\hline 

\end{tabular}
\end{table}

As shown in the fifth and eighth columns of above membership table for question 1.b, the statement "($\overline{A}$ $\cap$ $\overline{B}$)$\subseteq$(A $\cup$ $\overline{B}$) $\cap$ ($\overline{A}$ $\cup$ B)" is true.


\section*{Answer 2}
In order to prove this equation we need to show each side is a subset of the other. \\
Proving right side is a subset of the left side:

Since f has an inverse on the domains of $\textit{A}$x$\textit{C}$ and $\textit{B}$x$\textit{C}$, f is one-to-one and onto on these domains. Which means there must be a unique image for each element of $\textit{f}$ and likewise $\textit{f}^{-1}$. This allows us to do followings:  
\\
\\
$\textit{f}^{-1}$( ( A $\cap$ B ) x C) $\supseteq$ $\textit{f}^{-1}$ (A x C) $\cap$ $\textit{f}^{-1}$ (B x C) \\
$\textit{f}^{-1}$( ( A $\cap$ B ) x C) $\supseteq$ $\textit{f}^{-1}$ ( ( A x C) $\cap$ ( B x C ) ) ...(because f has inverse, explained at 2. paragraph) \\
$\textit{f}^{-1}$( ( A $\cap$ B ) x C) $\supseteq$ $\textit{f}^{-1}$ ( ( A $\cap$ B ) x C ) ...(by Distributive laws)
\\
\\
\\
\\
\\
Doing the same to prove left side is a subset of the right side: \\
$\textit{f}^{-1}$( ( A $\cap$ B ) x C) $\subseteq$ $\textit{f}^{-1}$ (A x C) $\cap$ $\textit{f}^{-1}$ (B x C) \\
$\textit{f}^{-1}$ ( ( A x C) $\cap$ ( B x C ) ) $\subseteq$ $\textit{f}^{-1}$ (A x C) $\cap$ $\textit{f}^{-1}$ (B x C) ...(by Distributive laws)\\
$\textit{f}^{-1}$ (A x C) $\cap$ $\textit{f}^{-1}$ (B x C) $\subseteq$ $\textit{f}^{-1}$ (A x C) $\cap$ $\textit{f}^{-1}$ (B x C)  ...(because f has inverse, explained at 2. paragraph) \\
\\
\\
So, the equation is true.



\section*{Answer 3}

$\textbf{a.}$
\\
$\textit{f(x)}$ = $\textit{$ln( x^{2} + 5)$}$ is not one-to-one since for x=5 $\in\Re$ and for x=-5 $\in\Re$ there is the same image $\textit{ln(30)}$
\\

$\textit{f(x)}$ = $\textit{ln( $x^{2}$ + 5)}$ is not onto since there is no element in domain so that $\textit{$ln( x^{2} + 5)$}$=0 $\Rightarrow$ ($x^{2}$=-4).
\\ \\ \\ \\
$\textbf{b.}$ 
\\
$\textit{f(x)}$ = $e^{e^{x^{7}}}$ is one-to-one since for an arbitrary a $\in\Re$, there is a unique image under f which is $e^{e^{a^{7}}}$ $\in\Re$ \\


$\textit{f(x)}$ = $e^{e^{x^{7}}}$ is not onto since for $e^{e^{x^{7}}}$ = 0 $\Rightarrow$ $x^{7}$ = log0, there is no element x $\in\Re$. \\


\section*{Answer 4}
$\textbf{Answer for 4.a}$ \\
There are three cases to consider: \\
(i)A and B are both finite, \\
(ii)A is infinite and B is finite,\\
(iii) A and B are both countably infinite.\\
\\
case(i): When A and B are both finite, AxB is also finite and therefore, countable. \\
csae(ii): Because A is countably infinite, its 	elements can be listed in an infinite sequence $a_{1}$, $a_{2}$, ... ,$a_{n}$, ... and because B is finite, its terms can be listed as 	$b_{1}$,$b_{2}$, ... ,$b_{m}$ for some positive integer m. We can list the elements of AxB as ($a_{1}$,$b_{1}$),($a_{1}$,$b_{2}$), ... ,($a_{1}$,$b_{m}$), ... ,($a_{n}$,$b_{1}$),($a_{n}$,$b_{2}$), ..., ($a_{n}$,$b_{m}$), ... \\
case(iii): Because A and B are both countably infinite, its elements can be listed in an infinite sequence. Their elements can be listed as $a_{1}$, $a_{2}$, ... ,$a_{n}$, ... and $b_{1}$, $b_{2}$, ... ,$b_{n}$, ... respectively. And we can list the elements of AxB by alternating and putting into tuples as ($a_{1}$,$b_{1}$), ($a_{1}$,$b_{2}$),($a_{2}$,$b_{1}$), ($a_{2}$,$b_{2}$), ($a_{1}$,$b_{3}$), ($a_{3}$,$b_{1}$), ($a_{2}$,$b_{3}$), ($a_{3}$,$b_{2}$), ($a_{3}$,$b_{3}$), ... \\
\\
\\
$\textbf{Answer for 4.b}$ \\
If A is uncountable and A$\subseteq$B, then B is uncountable. Suppose A represents all real numbers between (0,1). A is an uncountable set as proved in the textbook p.173-174. And suppose B represents all real numbers between (0,2). B has two parts: (0,1) and (1,2). As we said that the set of all real numbers between (0,1) is uncountable, B has this uncountable part too. So we say that a set with an uncountable subset is uncountable. So, if A is uncountable and A$\subseteq$B, then B is uncountable.
\\
\\
\\
$\textbf{Answer for 4.c}$ \\
If B is countable and A$\subseteq$B, then A is countable. Suppose B represents the positive integers such that there is a one-to-one correspondence function f(x)=x from $Z^{+}$ to B. And suppose A represents the positive even integers such that f(k)=t where t=2k k$\in$ $Z^{+}$. B can be listed as $b_{1}$=1, $b_{2}$=2, $b_{3}$=3, ... , $b_n$=n, ... and A can be listed as $a_{1}$=2, $a_{2}$=4, $a_{3}$=6, ... , $a_n$=2n, ... So we say that any countable subset of a countable set is countable. So, if B is countable and A$\subseteq$B, then A is countable.


\section*{Answer 5}
\textbf{Answer for 5.a} \\
To prove that we need to find a pair of witnesses C and k.
\textbf{Answer for 5.b} \\
To prove that we need to find a pair of witnesses C and k.
\section*{Answer 6}
\textbf{Answer for 6.a} \\
($3^{x}$-1)\textit{mod}($3^{y}$-1)=$3^{\textit{x mody}}$-1 \\
Using Theorem 4 in p.241 in textbook to write the left side differently; \\
($3^{x}$-1)\textit{mod}($3^{y}$-1) = $3^{x}$-1 + $k_{1}$*($3^{y}$-1)\\
Using Theorem 4 in p.241 in textbook to write the right side differently; \\
$3^{\textit{x mody}}$-1=$3^{x+k_{2}y}$-1\\
\\
$3^{x}$-1 + $k_{1}$*($3^{y}$-1) = $3^{x+k_{2}y}$-1 \\
The equation holds when $k_{1}$=0 and $k_{2}$=0. 





\textbf{Answer for 6.b} \\
277= 123*2 + 31 \\
123= 31*3 + 30 \\
31= 30*1 + 1 \\
30=1*30 + 0 \\
gcd(277,123)=gcd(123,31)=gcd(31,30)=1 \\




\end{document}

​

